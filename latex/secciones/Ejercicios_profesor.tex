\section{Ejercicios de clase}

\subsection*{Demostraciones usando diferencias finitas}

\[
\sum_{i=0}^{n} i = \frac{n(n+1)}{2}
\]

Sea \( a_n \) un polinomio, si la \( K \)-ésima diferencia finita de \( a_n \), \( \Delta^K a_n \), es constante, entonces \( a_n \) es un polinomio de grado \( K \).

\begin{itemize}
    \item Primera diferencia finita de \( a_n \):
    \[
    \Delta a_n
    \]
    \item Segunda diferencia finita de \( a_n \):
    \[
    \Delta(\Delta a_n) = \Delta^2 a_n
    \]
    \item Tercera diferencia finita de \( a_n \):
    \[
    \Delta(\Delta^2 a_n) = \Delta^3 a_n
    \]
    \item \( K \)-ésima diferencia finita de \( a_n \):
    \[
    \Delta(\Delta^{K-1} a_n) = \Delta^K a_n
    \]
\end{itemize}

\begin{table}[h]
    \centering
    \begin{tabular}{c|c|c|c|c|c}
        \( n \) & 0 & 1 & 2 & 3 & 4 \\
        \hline
        \( a_n = \sum_{i=0}^{n} i \) & 0 & 1 & 3 & 6 & 10 \\
        \hline
        \( \Delta a_n \) & \(1-0=1\) & \(3-1=2\) & \(6-3=3\) & \(10-6=4\) & \(15-10=5\) \\
        \hline
        \( \Delta^2 a_n \) & \(2-1=1\) & \(3-2=1\) & \(4-3=1\) & \(5-4=1\) &  \\
    \end{tabular}
    \caption{Diferencias finitas de \( a_n \)}
\end{table}

\[
\Delta a_n = a_{n+1} - a_n
\]

\[
\Delta^2 a_n = 1, \quad \forall n \in \mathbb{Z}
\]

Como \( \Delta^2 a_n \) es constante, entonces $\sum_{i=1}^{n} i$ es un polinomio de grado 2.

\[
a_n = \sum_{i=0}^{n} i = a n^2 + b n + c
\]

Si \( n = 0 \):

\[
a_0 = \sum_{i=0}^{0} i = 0
\]

\[
a_0 = a(0) + b(0) + c = 0
\]

\[
c = 0
\]


Si \( n = 1 \):

\[
a_1 = 1
\]

\[
a_1 = a \cdot 1^2 + b \cdot 1 + 0 = 1
\]

\[
a + b = 1
\]

Si \( n = 2 \):

\[
a_2 = 3
\]

\[
a_2 = a \cdot 2^2 + b \cdot 2 + 0 = 3
\]

\[
4a + 2b = 3
\]

\[
\begin{cases}
a + b = 1 \quad \text{(1)} \\
4a + 2b = 3 \quad \text{(2)}
\end{cases}
\]

De la ecuación (1):

\[
a = 1 - b
\]

De la ecuación (2):

\[
a = \frac{3 - 2b}{4} = \frac{3}{4} - \frac{b}{2}
\]

Igualando ambas expresiones para \( a \):

\[
1 - b = \frac{3}{4} - \frac{b}{2}
\]

\[
1 - \frac{3}{4} = b - \frac{b}{2}
\]

\[
\frac{1}{4} = \frac{b}{2}
\]

\[
b = \frac{1}{2}
\]

\begin{equation*}
    \begin{aligned}
    &\text{Usando la ecuación \textbf{(1)}} \quad a + b = 1 \quad \text{y} \quad b = \frac{1}{2} \\
    &a = 1 - \frac{1}{2} \\
    &a = \frac{1}{2}
    \end{aligned}
\end{equation*}

\[
a_n = a n^2 + b n + c ; \quad a = \frac{1}{2}, \quad b = \frac{1}{2}, \quad c = 0
\]

Cálculo de la sumatoria:

\[
\sum_{i=0}^{n} i = a_n = \frac{1}{2} n^2 + \frac{1}{2} n
\]

Factorizando:

\[
\sum_{i=0}^{n} i = \frac{n^2 + n}{2}
\]

Expresión final:

\[
\sum_{i=0}^{n} i = \frac{n(n+1)}{2}
\]

\section*{Demostración suma de los primeros enteros cuadrados}

\[
\sum_{i=1}^{n} i^2 = \frac{n(n+1)(2n+1)}{6}
\]

\begin{table}[h]
    \centering
    \begin{tabular}{c|c|c|c|c|c}
        \( n \) & 0 & 1 & 2 & 3 & 4 \\ \hline
        \( a_n = \sum_{i=0}^{n} i^2 \) & 0 & 1 & 5 & 14 & 30 \\ \hline
        \( \Delta a_n \) & 1 & 4 & 9 & 16 & 25 \\ \hline
        \( \Delta^2 a_n \) & 3 & 5 & 7 & 9 &  \\ \hline
        \( \Delta^3 a_n \) & 2 & 2 & 2 &  &  
    \end{tabular}
    \caption{Diferencias finitas de la suma de cuadrados}
    \label{tabla_suma_cuadrados}
\end{table}

\[
\Delta^3 a_n = 2 \Rightarrow \Delta^3 a_n \text{ es constante}
\]

\[
a_n = \sum_{i=0}^{n} i^2 = a n^3 + b n^2 + c n + d
\]

\subsection*{Determinación de coeficientes}

\textbf{Caso \( n = 0 \)}

\[
a_0 = \sum_{i=0}^{0} i^2 = 0
\]

\[
a_0 = a \cdot 0^3 + b \cdot 0^2 + c \cdot 0 + d = 0
\]

\[
d = 0
\]

\textbf{Caso \( n = 1 \)}

\[
a_1 = \sum_{i=0}^{1} i^2 = 1
\]

\[
a_1 = a \cdot 1^3 + b \cdot 1^2 + c \cdot 1 + d = 1
\]

\[
a + b + c = 1
\]

\textbf{Caso \( n = 2 \)}

\[
a_2 = \sum_{i=1}^{2} i^2 = 5
\]

\[
a_2 = 8a + 4b + 2c = 5
\]

\textbf{Caso \( n = 3 \)}

\[
a_3 = \sum_{i=1}^{3} i^2 = 14
\]

\[
a_3 = 27a + 9b + 3c = 14
\]

\subsection*{Sistema de ecuaciones}

\[
\begin{cases}
a + b + c = 1 \\
8a + 4b + 2c = 5 \\
27a + 9b + 3c = 14
\end{cases}
\]

\subsection*{Forma matricial}

\[
\begin{pmatrix}
1 & 1 & 1 & | 1 \\
8 & 4 & 2 & | 5 \\
27 & 9 & 3 & | 14
\end{pmatrix}
\]

\[
8F_1 - F_2 \rightarrow F_2
\]

\[
27F_1 - F_3 \rightarrow F_3
\]

\subsection*{Reducción de la matriz}

\[
\begin{pmatrix}
1 & 1 & 1 & | 1 \\
0 & 4 & 6 & | 3 \\
0 & 18 & 24 & | 13
\end{pmatrix}
\]

\[
\frac{1}{18} F_3 - \frac{1}{4} F_2 \rightarrow F_3
\]

\[
\begin{pmatrix}
1 & 1 & 1 & | 1 \\
0 & 4 & 6 & | 3 \\
0 & 0 & -\frac{1}{6} & | -\frac{1}{36}
\end{pmatrix}
\]

\subsection*{Resolviendo el sistema}

\[
a + b + c = 1
\]

\[
4b + 6c = 3
\]

\[
-\frac{1}{6} c = -\frac{1}{36}
\]

\[
c = \frac{1}{6}
\]

\[
b = \frac{3 - 6c}{4}
\]

\[
b = \frac{3 - 1}{4}
\]

\[
b = \frac{2}{4} = \frac{1}{2}
\]

\[
a = 1 - b - c
\]

\[
a = 1 - \frac{1}{2} - \frac{1}{6}
\]

\[
a = \frac{6}{6} - \frac{3}{6} - \frac{1}{6}
\]

\[
a = \frac{1}{3}
\]

\[
a_n = a n^3 + b n^2 + c n
\]

\[
a_n = \frac{1}{3} n^3 + \frac{1}{2} n^2 + \frac{1}{6} n
\]

\[
a_n = \frac{2}{6} n^3 + \frac{3}{6} n^2 + \frac{1}{6} n
\]

\[
a_n = \frac{2n^3 + 3n^2 + n}{6}
\]

\[
a_n = \frac{n(2n^2 + 3n + 1)}{6}
\]

\[
a_n = \frac{n \left( (n+1)(2n+1) \right)}{6}
\]

\[
a_n = \sum_{i=0}^{n} i^2 = \frac{n(n+1)(2n+1)}{6}
\]

