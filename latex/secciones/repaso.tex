\section{Repaso}

\begin{enumerate}
    \item Defina la integral definida y explique su interpretacion geométrica
\end{enumerate}

La integral definida se define como:
\[
\int_{a}^{b} f(x) \,dx = \lim_{n \to \infty} \sum_{i=1}^{n} f(x_i^*) \Delta x
\]
Donde $f(x)$ es continua en el intervalo \([a,b]\), que se divide en \(n\) subintervalos de igual longitud \(\Delta x\), y \(x_i^*\) es el punto medio entre \(x_{i-1}\) y \(x_i\).
La integral significa el área bajo la curva de la función \(f(x)\) en el intervalo \([a,b]\). Cuando n tiende a infinito, la suma de Riemann se convierte en el área exacta bajo la curva, ya que los rectángulos se hacen más pequeños y se ajustan mejor a la gráfica de la función.

\begin{enumerate}
    \setcounter{enumi}{1}
    \item Calcule la suma de Riemann izquierda para \( f(x) = x^2 \) en \([0,1]\) con \(n = 4\).
\end{enumerate}

\[
k = \frac{1}{4}
\]

\[
x_i = a + i\,k 
    = 0 + i \cdot \frac{1}{4} 
\]

\[
x_i = \frac{i}{4}
\]

\[
f(x_i) = \left(\frac{i}{4}\right)^2 
       = \frac{i^2}{16}
\]

Si la sumatoria es por la izquierda, entonces \(i\) empieza en \(0\) y termina en \(n - 1\). 

Si es por la derecha, entonces empieza en \(i = 1\) y termina en \(n\).

\[
k \sum_{i=0}^{n-1} f(x_i)
\]

\[
\frac{1}{4} \sum_{i=0}^{n-1} \frac{1}{16} i^2
\]

\[
\frac{1}{4} \cdot \frac{1}{16} \sum_{i=0}^{n-1} i^2
\]

\[
\sum_{i=1}^{n} i^2 = \frac{n(n+1)(2n+1)}{6}
\]

\[
\sum_{i=0}^{3} i^2 = \frac{12 \cdot 7}{6} = 14
\]

\[
\frac{1}{4} \cdot \frac{1}{16} \sum_{i=0}^{3} i^2 = \frac{1}{61} \cdot 14 = \frac{7}{32}
\]

\textbf{Comprobación sin fórmula de Gauss:}

\[
k (f(0k) + f(1k) + f(2k) + f(3k))
\]

\[
k [0^2 + k^2 + (2k)^2 + (3k)^2]
\]

\[
k [k^2 + 4k^2 + 9k^2]
\]

\[
k [14k^2] = 14k^3
\]

\[
= 14 \cdot \left(\frac{1}{4}\right)^3 = \frac{14}{64} = \frac{7}{32}
\]

\begin{enumerate}
    \setcounter{enumi}{2}
    \item Explique la diferencia entre una integral definida y una antiderivada
\end{enumerate}

La integral definida da como resultado un número correspondiente al área bajo la curva de la
función que se integra la antiderivada es el proceso contrario a la derivada
encuentran la función que modela el área bajo la curva, por ejemplo, la integral de la
velocidad en función del tiempo es la distancia, ya que la primera función es la derivada de la segunda.