\section{Ejercicios No Rutinarios}

\begin{enumerate}
    \item Diseñe un algoritmo para calcular la integral definida utilizando sumas de Riemann en Python.
\end{enumerate}



\begin{enumerate}
    \setcounter{enumi}{1}
    \item ¿Para qué funciones la suma de Riemann da una aproximación exavata de la integral? Justifique su respuesta.
\end{enumerate}

La suma de Riemman da una aproximacimación exacta de la integral para las
funciones constantes con n arbitrario, es decir si divido a dicho tipo de
función en $1$ o $n$ intervalos de rectángulos, el area va a ser la misma,
puesto que debajo de una linea recta puedo dibujar rectángulos perfectos.

Si tomo otro tipo de función y la divido en $n$ intervalos, entonces puedo hallar
el limite cuando $n$ tiende a infitito y esto me va a dar una aproximación exacta
de la integral, pues los rectángulos dibujados debajo de la curva son muy pequeños
y esto le da exactitud, pues cuando $n$ no es lo suficientemente grande, existen
rectángulos que quedan o muy por encima (aproximacimación por derecha) o muy por debajo de la curva (aproximacimación por izquierda).